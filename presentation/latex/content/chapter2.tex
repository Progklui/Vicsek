\subsection{Periodic Boundaries}

\begin{frame}
	\frametitle{2) Periodic Boundaries}
	\textbf{Trajectory.} Example.
	\begin{itemize}
	    \item $\rho = 400, v = 0.03, R = 0.01, D_{\text{rot}} = 0.01, \Delta t = 1.0$
	    \item Representative trajectory is shown
	    \item Equilibration in $\approx 200$ steps
	\end{itemize}
	\begin{figure}[H]
  		\includegraphics[width=\textwidth]{images/chapter2/flocks_N_20_L_1.000000_v_0.030000_R_0.010000_D_0.010000.png} 
  		%\caption*{Cantillano C., Grundpraktikum 2: Halbleiterbauelemente. Internal Proceedings. University of Innsbruck , 2021.}
	\end{figure}
\end{frame}

\begin{frame}
	\frametitle{2) Periodic Boundaries}
	\textbf{Configurations.} $R$-dependence.
	\begin{itemize}
	    \item $\rho = 400, v = 0.03, D_{\text{rot}} = 0.01, \Delta t = 1.0$
	    \item Bigger $R\Rightarrow$ bigger flocks 
	\end{itemize}
	\begin{figure}[H]
  		\includegraphics[width=\textwidth]{images/chapter2/N_20_L_1.000000_v_0.030000_R_D_0.010000.png} 
  		%\caption*{Cantillano C., Grundpraktikum 2: Halbleiterbauelemente. Internal Proceedings. University of Innsbruck , 2021.}
	\end{figure}
\end{frame}

\begin{frame}
	\frametitle{2) Periodic Boundaries}
	\textbf{Phase transitions.} Simulation settings.
	\begin{itemize}
	    \item $N_{\text{sim}} = 3000, N_{\text{eq}} = 1000, N_{\text{save}} = 1$ 
	    \item Fix $v$: vary
	    	\begin{itemize}
	    		\item $R$ and $D_{\text{rot}}$
	    		\item $\rho$ and $D_{\text{rot}}$
	    		\item $\rho$ and $R$
	    	\end{itemize}
	    \item Plot $\overline{v}_a$ against varying parameters
	\end{itemize}
\end{frame}

\begin{frame}
	\frametitle{2) Periodic Boundaries}
	\textbf{Phase transitions.} Comparing with reference.
	\begin{itemize}
	    \item Reference: $v = 0.03, R = 1.0, D_{\text{rot}} = 0.01, \Delta t = 1.0, \left(N,L\right) = \left(40,3.1\right), \left(100,5\right), \left(400,10\right), \left(4000,31.6\right), \left(10000,50\right)$
	    \item Our system: $v = 0.03$
	    % \item Trend behaviour is reproduced
	\end{itemize}
	\begin{figure}[H]
  		\includegraphics[width=\textwidth]{images/chapter2/va_against_eta_reference.png} 
  		\caption*{(left) Tamás Vicsek, András Czirók, Eshel Ben-Jacob, Inon Cohen, and Ofer Shochet. Novel Type of Phase Transition in a System of Self-Driven Particles. Phys. Rev. Lett. 75, 1226 – Published 7 August 1995}
	\end{figure}
\end{frame}

\begin{frame}
	\frametitle{2) Periodic Boundaries}
	\textbf{Phase transitions.} 2D Levels in parameter space.
	\begin{itemize}
	    \item $R$ against $\sqrt{2D_{\text{rot}}\Delta t}$
	\end{itemize}
	\begin{figure}[H]
  		\includegraphics[width=\textwidth]{images/chapter2/r_eta_transition_2D_plots_rho_comparison.png} 
  		%\caption*{Cantillano C., Grundpraktikum 2: Halbleiterbauelemente. Internal Proceedings. University of Innsbruck , 2021.}
	\end{figure}
\end{frame}

\begin{frame}
	\frametitle{2) Periodic Boundaries}
	\textbf{Phase transitions.} 2D Levels in parameter space.
	\begin{itemize}
	    \item $\rho$ against $\sqrt{2D_{\text{rot}}\Delta t}$
	\end{itemize}
	\begin{figure}[H]
  		\includegraphics[width=\textwidth]{images/chapter2/rho_eta_transition_2D_plots_r_comparison.png} 
  		%\caption*{Cantillano C., Grundpraktikum 2: Halbleiterbauelemente. Internal Proceedings. University of Innsbruck , 2021.}
	\end{figure}
\end{frame}

\begin{frame}
	\frametitle{2) Periodic Boundaries}
	\textbf{Phase transitions.} 2D Levels in parameter space.
	\begin{itemize}
	    \item $\rho$ against $R$
	\end{itemize}
	\begin{figure}[H]
  		\includegraphics[width=\textwidth]{images/chapter2/rho_r_transition_2D_plots_D_comparison.png} 
  		%\caption*{Cantillano C., Grundpraktikum 2: Halbleiterbauelemente. Internal Proceedings. University of Innsbruck , 2021.}
	\end{figure}
\end{frame}